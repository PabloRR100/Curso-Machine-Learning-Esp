\documentclass[]{article}
\usepackage{lmodern}
\usepackage{amssymb,amsmath}
\usepackage{ifxetex,ifluatex}
\usepackage{fixltx2e} % provides \textsubscript
\ifnum 0\ifxetex 1\fi\ifluatex 1\fi=0 % if pdftex
  \usepackage[T1]{fontenc}
  \usepackage[utf8]{inputenc}
\else % if luatex or xelatex
  \ifxetex
    \usepackage{mathspec}
  \else
    \usepackage{fontspec}
  \fi
  \defaultfontfeatures{Ligatures=TeX,Scale=MatchLowercase}
\fi
% use upquote if available, for straight quotes in verbatim environments
\IfFileExists{upquote.sty}{\usepackage{upquote}}{}
% use microtype if available
\IfFileExists{microtype.sty}{%
\usepackage{microtype}
\UseMicrotypeSet[protrusion]{basicmath} % disable protrusion for tt fonts
}{}
\usepackage[margin=1in]{geometry}
\usepackage{hyperref}
\hypersetup{unicode=true,
            pdftitle={Regresión Lineal Múltiple},
            pdfborder={0 0 0},
            breaklinks=true}
\urlstyle{same}  % don't use monospace font for urls
\usepackage{color}
\usepackage{fancyvrb}
\newcommand{\VerbBar}{|}
\newcommand{\VERB}{\Verb[commandchars=\\\{\}]}
\DefineVerbatimEnvironment{Highlighting}{Verbatim}{commandchars=\\\{\}}
% Add ',fontsize=\small' for more characters per line
\usepackage{framed}
\definecolor{shadecolor}{RGB}{248,248,248}
\newenvironment{Shaded}{\begin{snugshade}}{\end{snugshade}}
\newcommand{\KeywordTok}[1]{\textcolor[rgb]{0.13,0.29,0.53}{\textbf{#1}}}
\newcommand{\DataTypeTok}[1]{\textcolor[rgb]{0.13,0.29,0.53}{#1}}
\newcommand{\DecValTok}[1]{\textcolor[rgb]{0.00,0.00,0.81}{#1}}
\newcommand{\BaseNTok}[1]{\textcolor[rgb]{0.00,0.00,0.81}{#1}}
\newcommand{\FloatTok}[1]{\textcolor[rgb]{0.00,0.00,0.81}{#1}}
\newcommand{\ConstantTok}[1]{\textcolor[rgb]{0.00,0.00,0.00}{#1}}
\newcommand{\CharTok}[1]{\textcolor[rgb]{0.31,0.60,0.02}{#1}}
\newcommand{\SpecialCharTok}[1]{\textcolor[rgb]{0.00,0.00,0.00}{#1}}
\newcommand{\StringTok}[1]{\textcolor[rgb]{0.31,0.60,0.02}{#1}}
\newcommand{\VerbatimStringTok}[1]{\textcolor[rgb]{0.31,0.60,0.02}{#1}}
\newcommand{\SpecialStringTok}[1]{\textcolor[rgb]{0.31,0.60,0.02}{#1}}
\newcommand{\ImportTok}[1]{#1}
\newcommand{\CommentTok}[1]{\textcolor[rgb]{0.56,0.35,0.01}{\textit{#1}}}
\newcommand{\DocumentationTok}[1]{\textcolor[rgb]{0.56,0.35,0.01}{\textbf{\textit{#1}}}}
\newcommand{\AnnotationTok}[1]{\textcolor[rgb]{0.56,0.35,0.01}{\textbf{\textit{#1}}}}
\newcommand{\CommentVarTok}[1]{\textcolor[rgb]{0.56,0.35,0.01}{\textbf{\textit{#1}}}}
\newcommand{\OtherTok}[1]{\textcolor[rgb]{0.56,0.35,0.01}{#1}}
\newcommand{\FunctionTok}[1]{\textcolor[rgb]{0.00,0.00,0.00}{#1}}
\newcommand{\VariableTok}[1]{\textcolor[rgb]{0.00,0.00,0.00}{#1}}
\newcommand{\ControlFlowTok}[1]{\textcolor[rgb]{0.13,0.29,0.53}{\textbf{#1}}}
\newcommand{\OperatorTok}[1]{\textcolor[rgb]{0.81,0.36,0.00}{\textbf{#1}}}
\newcommand{\BuiltInTok}[1]{#1}
\newcommand{\ExtensionTok}[1]{#1}
\newcommand{\PreprocessorTok}[1]{\textcolor[rgb]{0.56,0.35,0.01}{\textit{#1}}}
\newcommand{\AttributeTok}[1]{\textcolor[rgb]{0.77,0.63,0.00}{#1}}
\newcommand{\RegionMarkerTok}[1]{#1}
\newcommand{\InformationTok}[1]{\textcolor[rgb]{0.56,0.35,0.01}{\textbf{\textit{#1}}}}
\newcommand{\WarningTok}[1]{\textcolor[rgb]{0.56,0.35,0.01}{\textbf{\textit{#1}}}}
\newcommand{\AlertTok}[1]{\textcolor[rgb]{0.94,0.16,0.16}{#1}}
\newcommand{\ErrorTok}[1]{\textcolor[rgb]{0.64,0.00,0.00}{\textbf{#1}}}
\newcommand{\NormalTok}[1]{#1}
\usepackage{graphicx,grffile}
\makeatletter
\def\maxwidth{\ifdim\Gin@nat@width>\linewidth\linewidth\else\Gin@nat@width\fi}
\def\maxheight{\ifdim\Gin@nat@height>\textheight\textheight\else\Gin@nat@height\fi}
\makeatother
% Scale images if necessary, so that they will not overflow the page
% margins by default, and it is still possible to overwrite the defaults
% using explicit options in \includegraphics[width, height, ...]{}
\setkeys{Gin}{width=\maxwidth,height=\maxheight,keepaspectratio}
\IfFileExists{parskip.sty}{%
\usepackage{parskip}
}{% else
\setlength{\parindent}{0pt}
\setlength{\parskip}{6pt plus 2pt minus 1pt}
}
\setlength{\emergencystretch}{3em}  % prevent overfull lines
\providecommand{\tightlist}{%
  \setlength{\itemsep}{0pt}\setlength{\parskip}{0pt}}
\setcounter{secnumdepth}{0}
% Redefines (sub)paragraphs to behave more like sections
\ifx\paragraph\undefined\else
\let\oldparagraph\paragraph
\renewcommand{\paragraph}[1]{\oldparagraph{#1}\mbox{}}
\fi
\ifx\subparagraph\undefined\else
\let\oldsubparagraph\subparagraph
\renewcommand{\subparagraph}[1]{\oldsubparagraph{#1}\mbox{}}
\fi

%%% Use protect on footnotes to avoid problems with footnotes in titles
\let\rmarkdownfootnote\footnote%
\def\footnote{\protect\rmarkdownfootnote}

%%% Change title format to be more compact
\usepackage{titling}

% Create subtitle command for use in maketitle
\newcommand{\subtitle}[1]{
  \posttitle{
    \begin{center}\large#1\end{center}
    }
}

\setlength{\droptitle}{-2em}
  \title{Regresión Lineal Múltiple}
  \pretitle{\vspace{\droptitle}\centering\huge}
  \posttitle{\par}
  \author{}
  \preauthor{}\postauthor{}
  \date{}
  \predate{}\postdate{}


\begin{document}
\maketitle

\section{Regresión Linear Múltiple
(R)}\label{regresion-linear-multiple-r}

La regresión linear utiliza el método de mínimos cuadrados para
encontrar la recta que resulta en la menor suma de errores al cuadrado
(RMSE: Root Mean Square Error). La palabra múltiple se refiere a que la
variable respuesta dependerá de más de 1 variable independiente: Y =
f(X1,\ldots{},Xn)

\subsection{Escenario del problema}\label{escenario-del-problema}

 Queremos encontrar la relación que existe entre un conjunto de
variables y el salario que podemos esperar tener cuando lo hayamos
conseguido. ¡Vamos a ello!

\begin{Shaded}
\begin{Highlighting}[]
\CommentTok{# 1. Importar librerías}
\KeywordTok{library}\NormalTok{(caTools)}
\KeywordTok{library}\NormalTok{(ggplot2)}
\end{Highlighting}
\end{Shaded}

\begin{Shaded}
\begin{Highlighting}[]
\CommentTok{# 2. Importar datos}
\NormalTok{datos <-}\StringTok{ }\KeywordTok{read.csv}\NormalTok{(}\StringTok{'../Datos/4.1.Empresas.csv'}\NormalTok{)}
\end{Highlighting}
\end{Shaded}

\begin{Shaded}
\begin{Highlighting}[]
\CommentTok{# 3. Codificcar variables categóricas}
\NormalTok{datos}\OperatorTok{$}\NormalTok{Pais <-}\StringTok{ }\KeywordTok{factor}\NormalTok{(datos}\OperatorTok{$}\NormalTok{Pais, }
                     \DataTypeTok{levels =} \KeywordTok{c}\NormalTok{(}\StringTok{'Nueva York'}\NormalTok{, }\StringTok{'California'}\NormalTok{, }\StringTok{'Florida'}\NormalTok{),}
                     \DataTypeTok{labels =} \KeywordTok{c}\NormalTok{(}\DecValTok{1}\NormalTok{, }\DecValTok{2}\NormalTok{, }\DecValTok{3}\NormalTok{))}
\NormalTok{datos}
\end{Highlighting}
\end{Shaded}

\begin{verbatim}
##    Investigacion Administracion Marketing Pais Beneficio
## 1      165349.20      136897.80 471784.10    1 192261.83
## 2      162597.70      151377.59 443898.53    2 191792.06
## 3      153441.51      101145.55 407934.54    3 191050.39
## 4      144372.41      118671.85 383199.62    1 182901.99
## 5      142107.34       91391.77 366168.42    3 166187.94
## 6      131876.90       99814.71 362861.36    1 156991.12
## 7      134615.46      147198.87 127716.82    2 156122.51
## 8      130298.13      145530.06 323876.68    3 155752.60
## 9      120542.52      148718.95 311613.29    1 152211.77
## 10     123334.88      108679.17 304981.62    2 149759.96
## 11     101913.08      110594.11 229160.95    3 146121.95
## 12     100671.96       91790.61 249744.55    2 144259.40
## 13      93863.75      127320.38 249839.44    3 141585.52
## 14      91992.39      135495.07 252664.93    2 134307.35
## 15     119943.24      156547.42 256512.92    3 132602.65
## 16     114523.61      122616.84 261776.23    1 129917.04
## 17      78013.11      121597.55 264346.06    2 126992.93
## 18      94657.16      145077.58 282574.31    1 125370.37
## 19      91749.16      114175.79 294919.57    3 124266.90
## 20      86419.70      153514.11      0.00    1 122776.86
## 21      76253.86      113867.30 298664.47    2 118474.03
## 22      78389.47      153773.43 299737.29    1 111313.02
## 23      73994.56      122782.75 303319.26    3 110352.25
## 24      67532.53      105751.03 304768.73    3 108733.99
## 25      77044.01       99281.34 140574.81    1 108552.04
## 26      64664.71      139553.16 137962.62    2 107404.34
## 27      75328.87      144135.98 134050.07    3 105733.54
## 28      72107.60      127864.55 353183.81    1 105008.31
## 29      66051.52      182645.56 118148.20    3 103282.38
## 30      65605.48      153032.06 107138.38    1 101004.64
## 31      61994.48      115641.28  91131.24    3  99937.59
## 32      61136.38      152701.92  88218.23    1  97483.56
## 33      63408.86      129219.61  46085.25    2  97427.84
## 34      55493.95      103057.49 214634.81    3  96778.92
## 35      46426.07      157693.92 210797.67    2  96712.80
## 36      46014.02       85047.44 205517.64    1  96479.51
## 37      28663.76      127056.21 201126.82    3  90708.19
## 38      44069.95       51283.14 197029.42    2  89949.14
## 39      20229.59       65947.93 185265.10    1  81229.06
## 40      38558.51       82982.09 174999.30    2  81005.76
## 41      28754.33      118546.05 172795.67    2  78239.91
## 42      27892.92       84710.77 164470.71    3  77798.83
## 43      23640.93       96189.63 148001.11    2  71498.49
## 44      15505.73      127382.30  35534.17    1  69758.98
## 45      22177.74      154806.14  28334.72    2  65200.33
## 46       1000.23      124153.04   1903.93    1  64926.08
## 47       1315.46      115816.21 297114.46    3  49490.75
## 48          0.00      135426.92      0.00    2  42559.73
## 49        542.05       51743.15      0.00    1  35673.41
## 50          0.00      116983.80  45173.06    2  14681.40
\end{verbatim}

\begin{Shaded}
\begin{Highlighting}[]
\CommentTok{# 3. Separar en Entrenamiento y Validación}
\KeywordTok{set.seed}\NormalTok{(}\DecValTok{123}\NormalTok{)}
\NormalTok{split <-}\StringTok{ }\KeywordTok{sample.split}\NormalTok{(datos}\OperatorTok{$}\NormalTok{Beneficio, }\DataTypeTok{SplitRatio =} \FloatTok{0.7}\NormalTok{)}
\NormalTok{entrenamiento <-}\StringTok{ }\KeywordTok{subset}\NormalTok{(datos, split}\OperatorTok{==}\OtherTok{TRUE}\NormalTok{)}
\NormalTok{validacion    <-}\StringTok{ }\KeywordTok{subset}\NormalTok{(datos, split}\OperatorTok{==}\OtherTok{FALSE}\NormalTok{)}
\NormalTok{train <-}\StringTok{ }\NormalTok{entrenamiento}
\NormalTok{test  <-}\StringTok{ }\NormalTok{validacion}
\end{Highlighting}
\end{Shaded}

\begin{Shaded}
\begin{Highlighting}[]
\CommentTok{# 4. Construir el Modelo}
\NormalTok{regresor <-}\StringTok{ }\KeywordTok{lm}\NormalTok{(}\DataTypeTok{formula =}\NormalTok{ Beneficio }\OperatorTok{~}\StringTok{ }\NormalTok{.,}
               \DataTypeTok{data =}\NormalTok{ train) }
\end{Highlighting}
\end{Shaded}

\begin{Shaded}
\begin{Highlighting}[]
\CommentTok{# 5. Hacer las prediciones para el conjunto de Validación}
\NormalTok{y_fit  <-}\StringTok{ }\KeywordTok{predict}\NormalTok{(regresor, }\DataTypeTok{newdata =}\NormalTok{ train)}
\NormalTok{y_pred <-}\StringTok{ }\KeywordTok{predict}\NormalTok{(regresor, }\DataTypeTok{newdata =}\NormalTok{ test) }
\end{Highlighting}
\end{Shaded}

\begin{Shaded}
\begin{Highlighting}[]
\CommentTok{# 7. Calcular el error}
\KeywordTok{library}\NormalTok{(Metrics)}
\NormalTok{y_real <-}\StringTok{ }\NormalTok{test}\OperatorTok{$}\NormalTok{Beneficio}
\NormalTok{RMSE <-}\StringTok{ }\KeywordTok{rmse}\NormalTok{(y_real, y_pred)}
\KeywordTok{print}\NormalTok{(RMSE)}
\end{Highlighting}
\end{Shaded}

\begin{verbatim}
## [1] 12062.72
\end{verbatim}

\begin{Shaded}
\begin{Highlighting}[]
\NormalTok{avg <-}\StringTok{ }\KeywordTok{mean}\NormalTok{(datos}\OperatorTok{$}\NormalTok{Beneficio)}
\KeywordTok{cat}\NormalTok{(}\StringTok{'RMSE is '}\NormalTok{, RMSE, }\StringTok{' over an average of: '}\NormalTok{, avg)}
\end{Highlighting}
\end{Shaded}

\begin{verbatim}
## RMSE is  12062.72  over an average of:  112012.6
\end{verbatim}


\end{document}
